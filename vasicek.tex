\documentclass[12pt]{article}
\usepackage{lingmacros}
\usepackage{amsmath}
\usepackage{xcolor}
\usepackage{tree-dvips}
\usepackage{hyperref}

\usepackage{geometry}
 \geometry{
 a4paper,
 total={170mm,257mm},
 left=20mm,
 top=20mm,
 }

\newcommand{\dd}{\mathop{}\!\text{d}}
\newcommand{\qBrownian}[1]{W(#1)}
\newcommand{\tBrownian}[1]{W^{T}(#1)}
\newcommand{\Brownian}[2]{W^{#1}(#2)}
\newcommand{\measure}[1][]{Q^{#1}}
\newcommand{\qMeasure}{\measure}
\newcommand{\TMeasure}{\measure[T]}
\newcommand{\sigmaP}{\sigma_P}
\newcommand{\sigmaf}{\sigma_f}
\newcommand{\sigmar}{\sigma_r}
\newcommand{\piterbargEq}[1]{Piterbarg Eq. #1}
\begin{document}

\title{Vasicek Model}
\maketitle

In risk-neutral measure $\qMeasure$, the money market account $\beta(t)$ is the associated
numeraire and $\qBrownian{t}$ is the adapted Brownian motion process. In $T$-forward measure $\TMeasure$,
zero coupon bond $P(t, T)$ is the associated numeraire and $\tBrownian{t}$ is the adapted
Brownian motion process.

\section{Zero Coupoun Bond Price Dynamics}
In risk-neutral measure, the deflated zero coupoun bond $P_{\beta}(t, T) = P(t, T)/\beta(t)$
is a martingale with lognormal distribution. The SDE is given as
\begin{equation}
    \dd P_\beta(t, T)  = - P_\beta(t, T) \sigma_P(t, T) \dd \qBrownian{t}
\end{equation}
where $\sigma_P(t,T)$ is the zero coupon bond volatility.

In risk-neutral measure, zero coupon bond $P(t, T)$ is a geometric brownian motion (GBM)
process and it is not a martingale.
(\piterbargEq{4.31})
\begin{equation}
    \dd P(t, T)/P(t, T) = r(t) \dd t - \sigma_P(t, T) \dd \qBrownian{t}
\end{equation}

In the risk-neutral measure, the forward zero coupon bond
$P(t, T_1, T_2) = P(t, T_2)/P(t, T_1)$ is a GBM process, and
it is not a martingale(\piterbargEq{4.32}).
The SDE is given as
\begin{align}
    \dd P(t, T_1, T_2)/P(t, T_1, T_2)  
    &= -\left[\sigmaP(t, T_2) - \sigma(t, T_1)\right] \sigmaP(t, T_1) \dd t \nonumber \\ 
    &\quad -\left[\sigmaP(t, T_2) - \sigmaP(t, T_1)\right] \dd \qBrownian{t}
\end{align}
In the $T$-forward measure, $P(t, T_1, T_2)$ is a martingale (\piterbargEq{4.33}).
The SDE is given as
\begin{align}
    \label{eq:fwd_bond_sde_T_measure}
    \dd P(t, T_1, T_2)/P(t, T_1, T_2)  
    &= -\left[\sigmaP(t, T_2) - \sigmaP(t, T_1)\right] \dd \tBrownian{t}
\end{align}

The relation between $\qBrownian{t}$ and $\tBrownian{t}$ is
\begin{equation}
    \label{eq:T_measure_relation}
    \dd \tBrownian{t} = \dd \qBrownian{t} + \sigmaP(t, T) \dd t
\end{equation}

\section{Forward Rate Dynamics}
By definition, the instantaneous forward rate is $f(t, T) = \frac{dP(t,T)}{dT}$.
In the risk-neutral measure, the process for $f(t,T)$ is normally distributed with
non-zero drift. The SDE is given as
\begin{align}
    \dd f(t,T) &= \sigmaf(t,T) \sigmaP(t, T) \dd t + \sigmaf(t,T)
        \dd \qBrownian{t} \\
    \dd f(t,T) &= \sigmaf(t,T) \left(\int_t^T \sigmaf(t,u) \dd u \right) \dd t + \sigmaf(t,T)
        \dd \qBrownian{t}
\end{align}
where 
\begin{align}
    \sigmaf(t,T) = \frac{\partial\sigmaP(t,T)}{\partial T}
\end{align}
    
In the $T$-forward measure, The process for $f(t,T)$ is a martingale with normal distribution
and the SDE is given as
\begin{align}
    \dd f(t,T) &= \sigmaf(t,T) \dd \tBrownian{t}
\end{align}

\section{Connection to Short-Rate Model}
From the instantaneous forward rate $f(t, T)$, the short rate $r(t) = f(t,t)$ is given as
\begin{equation}
    \label{eq:short_rate}
    r(t) = f(t,t) = f(0, t) + \int_0^t \sigmaf(u,t) \int_u^T
    \sigmaf(u,s) \dd s \dd u + \int_0^t \sigmaf(u, t) \dd \qBrownian{t}
\end{equation}
Consider the case where $\sigmaf(t,T)$ is a deterministic function with the special
choice
\begin{equation}
    \label{eq:gh}
    \sigmaf(t, T) = g(t) h(T)
\end{equation}
where $h(u)$ is a positive real function and $g(u)$ can take any sign.

This leads to the general Vasicek model, which gives the SDE of the short rate $r(t)$ as
\begin{equation}
    \dd r(t) = \left[\alpha(t) -\kappa(t)r(t)\right] \dd t + \sigmar(t) \dd \qBrownian{t}
\end{equation}
Note that the functions $\alpha(t)$, $\kappa(t)$ and $\sigmar(t)$ are not arbitrary.
They are all linked with Eq.\ref{eq:short_rate} through $g(u)$, $h(u)$ and the initial status,
which are given as
\begin{align}
    \alpha(t) &= \frac{\partial f(0, t)}{\partial t} + \kappa(t) f(0, t) +
        \int_0^t \sigmaf(u,t)\sigmaf(u,t)\dd u\\
    h(t) &= e^{-\int_0^t \kappa(u) \dd u} \\
    g(t) &= e^{\int_0^t \kappa(u) \dd u} \sigmar(t) \\
    \sigmaf(t, T) &= e^{-\int_t^T \kappa(u) \dd u } \sigmar(t)
\end{align}

Following \piterbargEq{10.19}, one can derive
the zero coupoun bond price $P(t,T)$, which is the main and foundamental result from Vasicek model.
One can follow \piterbargEq{10.19} to derive the following results.

First define the a list of quantities.
\begin{align}
    x(t) &= h(t) \int_0^t g(u)^2 \int_u^t h(s) \dd s \dd u + h(t) \int_0^t g(u) \dd \qBrownian{u} \\
    y(t) &= h(t)^2 \int_0^t g(u)^2 \dd u
\end{align}
The forward rate $f(t, T)$ is given as
\begin{align}
    f(t,T) &= f(0, T) + \frac{h(T)}{h(t)}
        \left(
            x(t) + \frac{y(t)}{h(t)} \int_t^T h(s) \dd s 
        \right)
\end{align}
The zero coupoun bond price $P(t, T)$ is given as
\begin{align}
    P(t, T) &= \exp \left( -\int_t^T f(t, u) \dd u \right) \\
        &=\exp \left( 
            -\int_t^T \left[ 
                f(0,u) + \frac{h(u)}{h(t)} \left(
                    x(t) + \frac{y(t)}{h(t)} \int_t^u h(s) \dd S
                    \right)
            \right] \dd u
        \right) \\
        &=\frac{P(0,T)}{P(0,t)} \exp \left[
            -\frac{x(t)}{h(t)} B(t,T) - \frac{y(t)}{h(t)^2} \int_t^T h(u) B(t, u) \dd u 
        \right] \\
        \label{eq:vol_term_1}
        &=\frac{P(0,T)}{P(0,t)} \exp \left[
            -\int_0^t g(u)^2 B(u, t) \dd u B(t,T) - V(t) \int_t^T h(u) B(t, u) \dd u
            -S(t) B(t,T)
        \right] \\
        \label{eq:vol_term_2}
        &=\frac{P(0,T)}{P(0,t)} \exp \left[
            -\int_0^t g(u)^2 B(u, t) \dd u B(t,T) - \frac{1}{2} V(t) B(t, T)^2 
            -S(t) B(t,T)
        \right] \\
        \label{eq:bond_price_start}
        &=\frac{P(0, T)}{P(0, t)} A(t,T) \exp\left[-B(t,T) S(t)\right]
\end{align}
where
\begin{align}
    A(t, T) &=\exp \left[
        -\int_0^t g(u)^2 B(u, t) \dd u B(t,T) - \frac{1}{2} V(t) B(t, T)^2 
    \right] \\
    B(t, T) &= \int_t^T h(u) \dd u \\
    S(t) &= \int_0^t g(u) \dd \qBrownian{u} \\
    \label{eq:bond_price_end}
    V(t) &= \int_0^t g(u)^2 \dd u
\end{align}
The equality from Eq.\ref{eq:vol_term_1} to Eq.\ref{eq:vol_term_2} is because
$\dd B(t,u) = h(u) \dd u$. $S(t)$ is called the state variable which introduces
the randomness. $V(t)$ is the variance of the state variable $S(t)$ up to time $t$.

\textbf{Eqs.\ref{eq:bond_price_start} - \ref{eq:bond_price_end} are the final expressions to be used
in the implementation for bond price under Vasicek model framework.
{\color{red}
Note here, the results are in the risk-neutral measure,
rather than the $T$-forward measure.
}
Using either risk-neutral measure or $T$-forward measure, the volatilities $\sigmaf(t,T)$,
$\sigmaP(t,T)$ are the foundamental quantities from Vasicek model that are eventually
expressed from $g(u)$ and $h(u)$ functions and used to derive the bond price $P(t, T)$ and
forward bond price $P(t, T_1, T_2)$. Therefore, functions $g(u)$ and $h(u)$ are conceptually
served as the Vasicek model parameters for calibration purpose.}
Below, the relations for these volatilities are summarized.
\begin{align}
    \sigmaf(t, T) &= g(t) h(T) \\
    \sigmaP(t, T) & = g(t) \int_t^T h(u) \dd u = g(t) B(t, T) \\
    \sigmaP(t, T_1, T_2) &= \sigmaP(t, T_1) - \sigmaP(t, T_2) = -g(t) B(T_1, T_2)
\end{align}

\section{Verification of P(t,T) bond price from two approaches}
As an execise, I am verifying that the expression of $P(T, T') = P(T,T,T')$ derived from
forward bond SDE (Eq.\ref{eq:fwd_bond_sde_T_measure}) is the same as the one
directly given by the Vasicek model for $P(T,T')$ from Eq.\ref{eq:vol_term_2}.

Under $T$-forward measure, the forward bond price $P(t, T, T')$
is a GBM process with SDE shown in Eq.\ref{eq:fwd_bond_sde_T_measure}. Therefore,
the expression for $P(t, T, T')$ is given as
\begin{align}
    P(t, T, T') &= P(0, T, T') \exp\left(
        \int_0^t -\frac{1}{2} \sigmaP(u, T, T')^2 \dd u +
            \int_0^t \sigmaP(u, T, T') {\color{red} \dd \tBrownian{u} }
        \right) \\
        &= P(0, T, T') \exp\left(
        \int_0^t -\frac{1}{2} g(u)^2 B(T,T')^2 \dd u -
            \int_0^t g(u) B(T,T')  \left[ 
                {\color{red}\dd \qBrownian{u}} + \sigmaP(u, T) \dd t
                \right]
        \right) \\
        &= \frac{P(0, T')}{P(0, T)} \exp \left(
        -\frac{1}{2} V(t) B(T,T')^2 - S(t) B(T,T') - \int_0^t g(u)^2 B(u, T) \dd u B(T, T')
        \right)
\end{align}
Therefore, let $t = T$ and the $P(T, T')$ is given as
\begin{align}
     P(T, T, T')  &= \frac{P(T, T')}{P(T, T)} = P(T, T') \\
     &= \frac{P(0, T')}{P(0, T)} \exp \left(
        -\frac{1}{2} V(T) B(T,T')^2 - S(T) B(T,T') - \int_0^T g(u)^2 B(u, T) \dd u B(T, T')
        \right)
\end{align}
This equation is identical to the results from Vasicek model by Eq.\ref{eq:vol_term_2}.

\end{document}