\documentclass[12pt]{article}
\usepackage{lingmacros}
\usepackage{amsmath,amssymb}
\usepackage{xcolor}
\usepackage{tree-dvips}
\usepackage{hyperref}
\usepackage{mathtools}
\usepackage[title]{appendix}

\usepackage{geometry}
 \geometry{
 a4paper,
 total={170mm,257mm},
 left=20mm,
 top=20mm,
 }

\DeclareMathOperator{\E}{\mathbb{E}}
\newcommand{\dd}{\mathop{}\!\text{d}}
\newcommand{\normaldist}{\mathcal{N}}
\newcommand{\qBrownian}[1]{W(#1)}
\newcommand{\tBrownian}[1]{W^{T}(#1)}
\newcommand{\Brownian}[2]{W^{#1}(#2)}
\newcommand{\measure}[1][]{Q^{#1}}
\newcommand{\qMeasure}{\measure}
\newcommand{\TMeasure}{\measure[T]}
\newcommand{\sigmaP}{\sigma_P}
\newcommand{\sigmaFP}{\tilde{\sigma}_P}
\newcommand{\sigmaf}{\sigma_f}
\newcommand{\sigmar}{\sigma_r}
\newcommand{\piterbargEq}[1]{Piterbarg Eq. #1}
\newcommand{\half}{\frac{1}{2}}
\begin{document}

\title{Vasicek Model: Thoery and Implementation Guidance}
\maketitle
\tableofcontents
\newpage

In risk-neutral measure $\qMeasure$, the money market account $\beta(t)$ is the associated
numeraire and $\qBrownian{t}$ is the adapted Brownian motion process. In $T$-forward measure $\TMeasure$,
zero coupon bond $P(t, T)$ is the associated numeraire and $\tBrownian{t}$ is the adapted
Brownian motion process.

\section{Zero Coupoun Bond Price Dynamics}
In risk-neutral measure, the deflated zero coupoun bond $P_{\beta}(t, T) = P(t, T)/\beta(t)$
is a martingale with lognormal distribution. The SDE is given as
\begin{equation}
    \dd P_\beta(t, T)  = - P_\beta(t, T) \sigma_P(t, T) \dd \qBrownian{t}
\end{equation}
where $\sigma_P(t,T)$ is the zero coupon bond volatility.

In risk-neutral measure, zero coupon bond $P(t, T)$ is a geometric brownian motion (GBM)
process and it is not a martingale. One can easily prove this by applying Ito
(see Appendix \ref{sec:bond_price_sde} and \piterbargEq{4.31}).
\begin{equation}
    \dd P(t, T)/P(t, T) = r(t) \dd t - \sigma_P(t, T) \dd \qBrownian{t}
\end{equation}

In the risk-neutral measure, the forward zero coupon bond
$P(t, T_1, T_2) = P(t, T_2)/P(t, T_1)$ is a GBM process, and
it is not a martingale. The SDE is given as
\begin{align}
    \label{eq:fwd_bond_sde_rn_measure}
    \dd P(t, T_1, T_2)/P(t, T_1, T_2)  
    &= -\left[\sigmaP(t, T_2) - \sigmaP(t, T_1)\right] \sigmaP(t, T_1) \dd t \nonumber \\ 
    &\quad -\left[\sigmaP(t, T_2) - \sigmaP(t, T_1)\right] \dd \qBrownian{t}
\end{align}
See Appendeix \ref{sec:fwd_bond_price_sde} and \piterbargEq{4.32}.

In the $T_1$-forward measure, $P(t, T_1, T_2)=\frac{P(t, T_2)}{P(t, T_1)}$
by its definition can be easily concluded to be a martingale (\piterbargEq{4.33}).
Therefore, the SDE under $T_1$-forward measure is given as
\begin{align}
    \label{eq:fwd_bond_sde_T_measure}
    \dd P(t, T_1, T_2)/P(t, T_1, T_2)  
    &= -\left[\sigmaP(t, T_2) - \sigmaP(t, T_1)\right] \dd W^{T_1}(t)
\end{align}

Based on Eq.~\ref{eq:fwd_bond_sde_rn_measure} and Eq.~\ref{eq:fwd_bond_sde_T_measure},
the relation between risk neutral measure and $T$-forward measure
($\qBrownian{t}$ and $\tBrownian{t}$) is
\begin{equation}
    \label{eq:T_measure_relation}
    \dd \tBrownian{t} = \dd \qBrownian{t} + \sigmaP(t, T) \dd t
\end{equation}

\section{Forward Rate Dynamics}
By definition, the instantaneous forward rate is $f(t, T) = -\frac{\dd \ln P(t,T)}{\dd T}$.
By Ito's lemma, in risk neutral measure,
\begin{align}
    \dd \ln P(t, T) = \mathcal{O}(\dd t) - \sigmaP(t, T) \dd \qBrownian{t}
\end{align}
Taking derivatives on both side w.r.t. $T$, we get the SDE for $f(t, T)$ as
\begin{align}
    \dd f(t, T) = \mu_f(t, T) \dd t + \sigmaf(t, T) \dd \qBrownian{t}
\end{align}
with
\begin{align}
    \sigmaf(t, T) = \frac{\partial}{\partial T} \sigmaP(t, T)
\end{align}

To determine the expression of the drift term $\mu_f(t, T)$ under risk neutral measure,
it would be easier to first consider the case under $T$-forward measure, then change
the measure back to risk neutral measure to obtain the analytical expression.

We need to use the property that the forward rate $L(t, T, T+\tau)$ is a martingale
under $(T+\tau)$-forward measure, that is,
\begin{align}
    L(t, T, T+\tau) = \E_t^{T+\tau}\left(L(u, T,T+\tau)\right), \quad t \leq u
\end{align}
Take the limit of $\tau$ going to zero, we have
\begin{align}
    f(t, T) = \E_t^T \left(f(u, T)\right), \quad t \leq u
\end{align}
Therefore, the forward rate $f(t, T)$ is a martingale under $T$-forward measure.
With the volatility of $\sigmaf(t, T)$ is given as above, we can conclude that the SDE
of $f(t, T)$ under $T$-forward measure must be
\begin{align}
    \dd f(t,T) &= \sigmaf(t,T) \dd \tBrownian{t}
\end{align}
Applying change of measure, the process of $f(t, T)$ under risk neutral measure
is given as
\begin{align}
    \dd f(t,T) &= \sigmaf(t,T) \sigmaP(t, T) \dd t + \sigmaf(t,T)
        \dd \qBrownian{t} \\
        \label{eq:fwd_rate_sde}
        &= \sigmaf(t,T) \left(\int_t^T \sigmaf(t,u) \dd u \right) \dd t + \sigmaf(t,T)
        \dd \qBrownian{t}
\end{align}
which indicates
\begin{equation}
    \mu_f(t, T) = \sigmaf(t,T) \left(\int_t^T \sigmaf(t,u) \dd u \right) 
\end{equation}
 
\section{Connection to Short-Rate Model}
From the instantaneous forward rate $f(t, T)$,
the short rate $r(t) = f(t,t)$ can be obatined by integrating on both side
of the Eq.~\ref{eq:fwd_rate_sde}
\begin{equation}
    \label{eq:short_rate}
    r(t) = f(t,t) = f(0, t) + \int_0^t \sigmaf(u,t) \int_u^T
    \sigmaf(u,s) \dd s \dd u + \int_0^t \sigmaf(u, t) \dd \qBrownian{t}
\end{equation}

Consider a special case where $\sigmaf(t,T)$ is a deterministic function
\begin{equation}
    \label{eq:gh}
    \sigmaf(t, T) = g(t) h(T)
\end{equation}
in which $h(u)$ is a positive real function and $g(u)$ can take any sign.
I have to emphasize here that Eq.~\ref{eq:gh} is a very important assumption, because
it will eventually lead to the general Vasicek model that we are familiar with
(as shown in Piterbarg's book section 4.5.2). The SDE of the short rate $r(t)$
for the general Vasicek model is
\begin{equation}
    \label{eq:vasicek_sde}
    \dd r(t) = \left[\alpha(t) -\kappa(t)r(t)\right] \dd t + \sigmar(t) \dd \qBrownian{t}
\end{equation}
Note, because of the connection between Eq.~\ref{eq:gh} and Eq.~\ref{eq:vasicek_sde},
the functions $\alpha(t)$, $\kappa(t)$ and $\sigmar(t)$ in Vasicek model are not arbitrary.
They are all linked with Eq.\ref{eq:short_rate} through $g(u)$, $h(u)$ and the initial status.
The analytical expression are given as below
\begin{align}
    \alpha(t) &= \frac{\partial f(0, t)}{\partial t} + \kappa(t) f(0, t) +
        \int_0^t \sigmaf(u,t)\sigmaf(u,t)\dd u\\
    h(t) &= e^{-\int_0^t \kappa(u) \dd u} \\
    g(t) &= e^{\int_0^t \kappa(u) \dd u} \sigmar(t) \\
    \sigmaf(t, T) &= e^{-\int_t^T \kappa(u) \dd u } \sigmar(t)
\end{align}

\section{Vasicek Model Application}
\subsection{Zero Coupoun Bond Price}
Following \piterbargEq{10.19}, one can derive the zero coupoun bond price $P(t,T)$
\begin{align}
    \label{eq:bond_price}
    P(t, T) &=\frac{P(0, T)}{P(0, t)} A(t,T) \exp\left[-B(t,T) S(t)\right]
\end{align}
where
\begin{align}
    \label{eq:A}
    A(t, T) &=\exp \left[
        -\int_0^t g(u)^2 B(u, t) \dd u B(t,T) - \frac{1}{2} V(t) B(t, T)^2 
    \right] \\
    \label{eq:B}
    B(t, T) &= \int_t^T h(u) \dd u \\
    \label{eq:S}
    S(t) &= \int_0^t g(u) \dd \qBrownian{u} \\
    \label{eq:V}
    V(t) &= \int_0^t g(u)^2 \dd u
\end{align}
$S(t)$ is called the state variable which introduces
the randomness. $V(t)$ is the variance of the state variable $S(t)$ up to time $t$.
The detailed derivation of Eq.~\ref{eq:bond_price}
can be found in Appendix~\ref{sec:vasicek_bond_price_derivation}.
Details for the implementation of evaluating $A(t, T)$ and $B(t, T)$
can be found in Appendix~\ref{sec:vasicek_AB_implementation}

The bond price (Eqs.\ref{eq:bond_price} - \ref{eq:V})
are the most foundamental results from Vasicek model.
From the bond price $P(t, T)$, one can derive the forward bond price
$P(t, T_1, T_2)$ and the forward simple rate $L(t, T_1, T_2)$.
Eqs.\ref{eq:bond_price} - \ref{eq:V} are the final expressions to be used
in the implementation for bond price under Vasicek model framework.
{\color{red}
Note here, the results are in the risk-neutral measure,
NOT in the $T$-forward measure.}
If you want to use the bond price in forward measure, remember to change
the measure first (See Appendix~\ref{sec:bond_price_change_of_measure_verification}
for related topic and comment).

No matter of using risk-neutral measure or $T$-forward measure,
the volatilities $\sigmaf(t,T)$, $\sigmaP(t,T)$ do not change and
they are the foundamental quantities from Vasicek model which are 
expressed by $g(u)$ and $h(u)$ functions and used to derive the bond price $P(t, T)$ and
forward bond price $P(t, T_1, T_2)$. Therefore, functions $g(u)$ and $h(u)$ are conceptually
served as the Vasicek model parameters for calibration purpose.
Below, the relations for these volatilities are summarized.
\begin{align}
    \sigmaf(t, T) &= g(t) h(T) \\
    \nonumber
    \sigmaP(t, T) & = g(t) \int_t^T h(u) \dd u = g(t) B(t, T) \\
    \sigmaFP(t, T_1, T_2) &= \sigmaP(t, T_2) - \sigmaP(t, T_1) = g(t) B(T_1, T_2)
\end{align}

\subsection{Simple Averaging Caplet Price}
For simple averaging caplet, there is only one cash flow
paying at the end of the period, but the interest rate is 
calculated from the mean averaging of the a sequence of small
rates. Denote $T_1, T_2, T_3, \cdots, T_N$
as the dates within the period to give the $N$
number of small rates $L(t, T_i, T_{i+1})$ for 
for each of the interest accrual interval $[T_i, T_{i+1}]$.
Depending on when the rate is fixed, there could be 
two types of averaging caplet. If all the rates are
fixed at the begining (meaning all fixed at $T_1$),
it is the forward looking caplet. If all the rates
are fixed at the $T_i$ for each interval accordingly,
it is the backward looking caplet. The approach to
evalulate the price of these two types of averaging
caplets are different. The discussion here also applies
to the similiar cases for compounding caplet with forward or
backward looking types.

Note the payment is occurred at the end of the period
$T_N$. It would be easier to price using $T_N$-forward
measure.
Under $T_N$-forward measure, the price of the
backward caplet at current time $t$ is 
\begin{equation}
    V^{b}(t) = P(t, T_N) \E^{T_N} \left[
        \frac{1}{N} \sum_{i}^{N} L(T_i, T_i, T_{i+1}) - K
    \right]^+
\end{equation}
The price of the forward caplet at current time $t$ is 
\begin{equation}
    V^{f}(t) = P(t, T_N) \E^{T_N} \left[
        \frac{1}{N} \sum_{i}^{N} L(T_1, T_i, T_{i+1}) - K
    \right]^+
\end{equation}
{\color{red} In both cases, we need the evoluation of the the forward
simple rate $L$ under $T_N$-forward measure.} The only difference is what time the forward
rate is evulated at. Below we take the backward case as an 
example, and similiar derviation can be easily applied to the
forward case.

For backward caplet, each small rate $L(T_i, T_i, T_{i+1})$ by its definition is
connected with the forward band price $P(T_i, T_i, T_{i+1})$
at time $T_i$ accordingly,
\begin{align}
    L(T_i, T_i, T_{i+1}) = \left(
        \frac{1}{P(T_i, T_i, T_{i+1})} - 1
    \right) \frac{1}{\tau_i}
\end{align}
Therefore, we need to simulate the evolution of the
forwad bond $P(t, T_i, T_{i+1})$ under $T_N$-forward
measure. Apply changing of measure, we have
\begin{align}
   \dd W^{T_1}(t) &= \dd W(t) + \sigmaP(t, T_1) \dd t \\
   \dd W^{T_2}(t) &= \dd W(t) + \sigmaP(t, T_2) \dd t \\
   \dd W^{T_1}(t) &= \dd W^{T_2}(t) - \left[ \sigmaP(t, T_2) - \sigmaP(t, T_1) \right] \dd t\\
    &= \dd W^{T_2} - \sigmaFP(t, T_1, T_2) \dd t
\end{align}
Switch from $T_1$-forwad measure to $T_N$-forward measure,
the SDE of $P(t, T_1, T_2)$ is given as
\begin{align}
    \frac{\dd P(t, T_1, T_2)}{P(t, T_1, T_2)}
        &= -\sigmaFP(t, T_1, T_2) \dd W^{T_1} \\
        &= -\sigmaFP(t, T_1, T_2) \sigmaFP(t, T_1, T_N) \dd t - \sigmaFP(t, T_1, T_2) \dd W^{T_N}
\end{align}
This is a GBM Process and the solution of $P(t, T_1, T_2)$
under $T_N$-forwad measure is given as
\begin{align}
    P(t, T_1, T_2) &= \frac{P(0, T_2)}{P(0, T_1)}
    \exp \left\{
        \int_0^t \left[
            \sigmaFP(s, T_1, T-2) \sigmaFP(s, T_1, T_N)
            - \half \sigmaFP(s, T_1, T_2)^2
        \right] \dd s \right. \\
        &\left. \quad
            - \int_0^t \sigmaFP(s, T_1, T_2) \dd W^{T_N}(s)
    \right\} \\
    &=\frac{P(0, T_2)}{P(0, T_1)} \exp \left[Y(t)\right]
\end{align}
where $Y(t)$ is a random variable within the big bracket,
and it is a quantity under $T_N$-forward measure,
\begin{align}
    Y(t) &=
         \int_0^t \left[
            \sigmaFP(s, T_1, T-2) \sigmaFP(s, T_1, T_N)
            - \half \sigmaFP(s, T_1, T_2)^2
        \right] \dd s 
            - \int_0^t \sigmaFP(s, T_1, T_2) \dd W^{T_N}(s)
\end{align}
with drift $\mu_{Y}(t)$ and variance $\sigma_{Y}(t)$.
By some arrangement, we can derive the drfit and variance
as below,
\begin{align}
    \nonumber
    \mu_{Y}(t) &=
        \int_0^t \left[
            \sigmaFP(s, T_1, T_2) \sigmaFP(s, T_1, T_N)
            - \half \sigmaFP(s, T_1, T_2)^2
        \right] \dd s \\
    \nonumber
    &= \int_0^t g(s)^2 \dd s \cdot B(T_1, T_2) B(T_1, T_N)
    - \half \int_0^t g(s)^2 \dd s \cdot B(T_1, T_2)^2 \\
    \nonumber
    &= V(t) B(T_1, T_2) B(T_1, T_N) - \half V(t) B(T_1, T_2)^2 \\
    \nonumber
    &= V(t) B(T_1, T_2) \left[B(T_1, T_N) - \half B(T_1, T_2)\right]\\
    \nonumber
    &= V(t) B(T_1, T_2) \left( B(T_1, T_N) - \half
        \left[ B(T_1, T_N) - B(T_2, T_N) \right] \right) \\
    &= \half V(t) B(T_1, T_2) \left[B(T_1, T_N) + B(T_2, T_N) \right] \\
    \sigma_{Y}(t) &=\sqrt{V(t)} B(T_1, T_2) \\
    V(t) &= \int_0^t g(s)^2 \dd s
\end{align}
Now we can express each small rate $L(T_i, T_i, T_{i+1})$
as the function of the random variable $Y(T_i)$ as
\begin{align}
    \nonumber
    L(T_i, T_i, T_{i+1}) &= \left(
        \frac{1}{P(T_i, T_i, T_{i+1})} - 1
    \right) \frac{1}{\tau_i} \\
    &= \left[
        \frac{P(0, T_i)}{P(0, T_{i+1})} \exp(-Y(T_i)) - 1
    \right] \frac{1}{\tau_i} \\
    &= h\left(Y(T_i)\right)
\end{align}
where $Y(T_i) \sim \normaldist\left(\mu_Y(T_i), \sigma_{Y}(T_i)\right)$
under $T_N$-forward measure. For short notation,
we denote $L$ is a function of $Y$, $L = h(Y)$.

To calculate the expectation of $L(T_i, T_i, T_{i+1})$
under $T_N$-forward measure, we can use gauss-hermit quadrature.
If $Y\sim \normaldist(\mu, \sigma)$, the expectation can be
approximate as
\begin{align}
    \E^{T_N} \left[h(Y)\right] 
        &= \int_{-\infty}^{+\infty}
            \frac{1}{\sigma \sqrt{2\pi}} \exp\left(
                -\frac{(y-\mu)^2}{2\sigma^2}
            \right) h(y) \dd y \\
        &= \frac{1}{\sqrt{\pi}} \int_{-\infty}^{+\infty}
            \exp(-x^2) h(\sqrt{2} \sigma x + \mu) \dd x \\
        &\approx \sum_i^n \frac{1}{\sqrt{\pi}} \omega_{i} h(\sqrt{2}\sigma x_i + \mu)
\end{align}
where $X\sim \normaldist(0,1)$, $\omega_{i}$ is the gauss-hermit
coefficients, and $Y$ can be expressed from $X$ as
$ Y = \sqrt{2}\sigma X + \mu$.

Therefore, in the case of forward rate $L(T_i, T_i, T_{i+1})$,
we have the random variable $-Y(T_i)\sim \normaldist(-\mu_Y(T_i),
\sigma_Y(T_i))$. The random forward rate is expressed as
\begin{align}
    \nonumber
   L(T_i, T_i, T_{i+1}; X) 
   &= \left[
    \frac{P(0, T_i)}{P(0, T_{i+1})} \exp\left[-Y(T_i)\right] 
    - 1
   \right]   \frac{1}{\tau_i} \\
   &= \left[
    \frac{P(0, T_i)}{P(0, T_{i+1})} \exp
    \left[\sigma_Y(T_i) \sqrt{2} X - \mu_{Y}(T_i)\right] 
    - 1
   \right]   \frac{1}{\tau_i} \\
\end{align}

To evaluate the caplet price, we can use numerical
integration (like gauss-hermit quadrature)
to calculate the expectation. Note the options payoff
function is non-linear and there are multiple simple rates
used in the expectation. This leads the exact expectation
evulation to be a multi-dimensional integration. However,
in practice, we usually approximate the multi-dimensional
integraion into a single dimensional integraion.

\newpage
\appendix
\begin{appendices}
\section{Detailed derivations}
\subsection{Bond price SDE} \label{sec:bond_price_sde}
\begin{equation}
    \dd P(t, T)/P(t, T) = r(t) \dd t - \sigma_P(t, T) \dd \qBrownian{t}
\end{equation}
Proof:
\begin{align}
    P(t, T) &= P_\beta(t, T) \cdot \beta(t) \\
    \nonumber
    \dd P(t, T) &= \dd P_\beta \cdot \beta + P_\beta \dd \beta
        + \dd P_\beta \cdot \dd \beta \\
        \nonumber
        &= -\sigmaP P_\beta \beta \dd W + P_\beta r\beta \dd t \\
        &= -\sigmaP P \dd W + P r \dd t
\end{align}

\subsection{Forward bond price SDE} \label{sec:fwd_bond_price_sde}
\begin{align}
    \frac{\dd P(t, T_1, T_2)}{P(t, T_1, T_2)}
    &= -\left[\sigmaP(t, T_2) - \sigmaP(t, T_1)\right] \sigmaP(t, T_1) \dd t 
    -\left[\sigmaP(t, T_2) - \sigmaP(t, T_1)\right] \dd \qBrownian{t}
\end{align}
There are two ways to prove the above relation.

\subsubsection*{Approach 1: using GBM property}
$P(t, T)$ is a GBM process.
\begin{equation}
    P(t, T) = P(0, T) \exp \left[
        (r(t) - \half \sigmaP^2)t - \sigmaP W(t)
    \right]
\end{equation}
Therefore, we can write down the expression for the forward bond price.
\begin{align}
    \nonumber
    \tilde{P} &= \frac{P_2}{P_1} \\
    \nonumber
        &=\frac{P(0, T_2)}{P(0, T_1)} \exp\left[
            -\half (\sigma_2^2 - \sigma_1^2) t - (\sigma_2 - \sigma_1) W
        \right] \\
        &=\tilde{P}(0, T_1, T_2) \exp \left[
            \left(
                -(\sigma_2 - \sigma_1) \sigma_1 - \half (\sigma_2 - \sigma_1)^2
            \right) t - (\sigma_2 - \sigma_1) W
        \right]
\end{align}
From the above equation, we can back out the SDE for the GBM process as
\begin{equation}
    \frac{\dd \tilde{P}}{\tilde{P}} = -(\sigma_2 - \sigma_1) \sigma_1 \dd t 
        - (\sigma_2 - \sigma_1) \dd W(t)
\end{equation}

\subsubsection*{Approach 2: using Ito}
\begin{align}
    \tilde{P} &= \frac{P_2}{P_1} \\
    \dd \tilde{P}  &= \dd P_2 \frac{1}{P_1} + P_2 \dd \frac{1}{P_1} + \dd P_2 \dd \frac{1}{P_1} \\
    \dd P &= P (r \dd t - \sigma \dd W) \\
    \nonumber
    \dd \frac{1}{P} &= - \frac{1}{P^2} \dd P + \frac{1}{P^3} (\dd P)^2 \\
        \nonumber
        &= -\frac{1}{P} (r \dd t - \sigma \dd W) + \frac{\sigma^2}{P} \dd t\\ 
        &= \frac{1}{P} \left[
            (\sigma^2 - r) \dd t + \sigma \dd W
        \right]
\end{align}
Substitute everything into the definition, we can get 
\begin{align}
    \nonumber
    \dd \tilde{P} &= \frac{P_2}{P_1}(r \dd t - \sigma_2 \dd W) +
        \frac{P_2}{P_1} \left[
            (\sigma_1^2 - r ) \dd t + \sigma_1\dd W
        \right] - \frac{P_2}{P_1} \sigma_1 \sigma_2 \dd t \\
        \nonumber
        &= \frac{P_2}{P_1} \left[
            (\sigma_1 - \sigma_2) \sigma_1 \dd t - (\sigma_2 - \sigma_1) \dd W
        \right] \\
        &= \tilde{P} \left[
            (\sigma_1 - \sigma_2) \sigma_1 \dd t - (\sigma_2 - \sigma_1) \dd W
        \right]
\end{align}

\label{sec:vasicek_bond_price_derivation}
\subsection{Vasicek model bond price analytical expression derivation}
Use Eq.~\ref{eq:gh} and integrate Eq.~\ref{eq:fwd_rate_sde}, the forward rate
$f(t, T)$ is
\begin{align}
    \nonumber
    f(t, T) &= f(0, T) + \int_0^t g(u) h(T) \int_u^T g(u) h(s) \dd s \dd u
        + \int_0^t g(u) h(T) \dd W(u) \\
        &= f(0, T) + h(T) \int_0^t g(u)^2 \int_u^T  h(s) \dd s \dd u
        + h(T)\int_0^t g(u)  \dd W(u) 
\end{align}
Define the following auxiliary quantities
\begin{align}
    x(t) &= h(t) \int_0^t g(u)^2 \int_u^t h(s) \dd s \dd u + h(t) \int_0^t g(u) \dd \qBrownian{u} \\
    y(t) &= h(t)^2 \int_0^t g(u)^2 \dd u
\end{align}
The forward rate $f(t, T)$ can be rewritten as
\begin{align}
    f(t,T) &= f(0, T) + \frac{h(T)}{h(t)}
        \left(
            x(t) + \frac{y(t)}{h(t)} \int_t^T h(s) \dd s 
        \right)
\end{align}
The zero coupoun bond price $P(t, T)$ is given as
\begin{align}
    P(t, T) &= \exp \left( -\int_t^T f(t, u) \dd u \right) \\
        &=\exp \left( 
            -\int_t^T \left[ 
                f(0,u) + \frac{h(u)}{h(t)} \left(
                    x(t) + \frac{y(t)}{h(t)} \int_t^u h(s) \dd s
                    \right)
            \right] \dd u
        \right) \\
        &=\frac{P(0,T)}{P(0,t)} \exp \left[
            -\frac{x(t)}{h(t)} B(t,T) - \frac{y(t)}{h(t)^2} \int_t^T h(u) B(t, u) \dd u 
        \right] \\
        \label{eq:vol_term_1}
        &=\frac{P(0,T)}{P(0,t)} \exp \left[
            -\int_0^t g(u)^2 B(u, t) \dd u B(t,T) - V(t) \int_t^T h(u) B(t, u) \dd u
            -S(t) B(t,T)
        \right] \\
        \label{eq:vol_term_2}
        &=\frac{P(0,T)}{P(0,t)} \exp \left[
            -\int_0^t g(u)^2 B(u, t) \dd u B(t,T) - \frac{1}{2} V(t) B(t, T)^2 
            -S(t) B(t,T)
        \right] \\
        \label{eq:bond_price_start}
        &=\frac{P(0, T)}{P(0, t)} A(t,T) \exp\left[-B(t,T) S(t)\right]
\end{align}
The equality from Eq.\ref{eq:vol_term_1} to Eq.\ref{eq:vol_term_2} is because
$\dd B(t,u) = h(u) \dd u$. 

\label{sec:bond_price_change_of_measure_verification}
\subsection{Verification of P(T,T') bond price from two approaches}
For bond price $P(T, T') = P(T, T, T') = \frac{P(T, T')}{P(T, T)}$, there are two
ways to calculate its value. One is directly from Vasicek model by
using Eq.~\ref{eq:bond_price}, and the other is from the forward bond price
$\left. P(t, T, T') \right \vert_{t=T}$ by using the forward bond price is a GBM process
(Eq.~\ref{eq:fwd_bond_sde_rn_measure} and ~\ref{eq:fwd_bond_sde_T_measure}).
As an execise, I am verifying the two approaches
agree with each other.

Based on the GBM process of forwad bond (Eq.~\ref{eq:fwd_bond_sde_rn_measure} and
~\ref{eq:fwd_bond_sde_T_measure}), the expression for $P(t, T, T')$ is given as
\begin{align}
    P(t, T, T') &= P(0, T, T') \exp\left(
        \int_0^t -\frac{1}{2} \sigmaP(u, T, T')^2 \dd u +
            \int_0^t \sigmaP(u, T, T') {\color{red} \dd \tBrownian{u} }
        \right) \\
        &= P(0, T, T') \exp\left(
        \int_0^t -\frac{1}{2} g(u)^2 B(T,T')^2 \dd u -
            \int_0^t g(u) B(T,T')  \left[ 
                {\color{red}\dd \qBrownian{u}} + \sigmaP(u, T) \dd t
                \right]
        \right) \\
        &= \frac{P(0, T')}{P(0, T)} \exp \left(
        -\frac{1}{2} V(t) B(T,T')^2 - S(t) B(T,T') - \int_0^t g(u)^2 B(u, T) \dd u B(T, T')
        \right)
\end{align}
Evaulate the forwad bond at time $t = T$, the $P(T, T')$ is given as
\begin{align}
     P(T, T, T')  &= \frac{P(T, T')}{P(T, T)} = P(T, T') \\
     &= \frac{P(0, T')}{P(0, T)} \exp \left(
        -\frac{1}{2} V(T) B(T,T')^2 - S(T) B(T,T') - \int_0^T g(u)^2 B(u, T) \dd u B(T, T')
        \right)
\end{align}
This equation is identical to the results from Vasicek model by Eq.\ref{eq:bond_price}.

\label{sec:vasicek_AB_implementation}
\subsection{Implementation of Vasicek A and B term for PWC}
In the case of $g(u)$ and $h(u)$ being peice-wise constant (PWC) function,
define
\begin{align}
    g(u) =
    \begin{cases}
        g_1, & t_0 < u < t_1 \\
        g_2, & t_1 < u < t_2 \\
        \vdots & \\
        g_N, & t_{N-1} < u < t_N \\
    \end{cases}
\end{align}
Similiar definition for $h(u)$ function.
Let $t_i < t < t_{i+1}$, define
\begin{align}
    \Delta t_i &\coloneqq t - t_i \\
    \delta t_i &\coloneqq t_i - t_{i-1} \\
    G_i &\coloneqq \int_0^{t_i} g(u)^2 \dd u = \sum_0^{i-1} g_k^2 (t_{k+1} - t_k) \\
    H_i &\coloneqq \int_0^{t_i} h(u) \dd u = \sum_0^{i-1} h_k^2 (t_{k+1} - t_k) \\
    \Delta G_i &\coloneqq \int_{t_i}^t g(u)^2 \dd u = g_i^2 \Delta t_i \\
    \Delta H_i &\coloneqq \int_{t_i}^t h(u) \dd u = h_i^2 \Delta t_i
\end{align}

\subsubsection{Expression of A(t, T)}
\begin{align}
    \label{Aeq:A}
    A(t, T) = \exp \left[
        - \int_0^t g(u)^2 B(u, t) \dd u B(t, T) - \half V(t) B(t, T)^2
    \right]
\end{align}

We derive the first term in Eq.~\ref{Aeq:A} here. The second term is trivial to
implement. Rewrite the integration on $h(u)$ into two parts,
\begin{align}
    \nonumber
    \int_0^t g(u)^2 B(u, t) \dd u &= \int_0^t g(u)^2 \int_u^t h(s) \dd s \dd u \\
        \nonumber
        &= \int_0^t g(u)^2 \int_0^t h(s) \dd s \dd u 
            - \int_0^t g(u)^2 \int_0^u h(s) \dd s \dd u \\
        \label{Aeq:A_pwc_2nd_term}
        &=(G_i + \Delta G_i) (H_i + \Delta H_i) 
            - \int_0^t g(u)^2 \int_0^u h(s) \dd s \dd u 
\end{align}
Next, we evaluate the second term in Eq.~\ref{Aeq:A_pwc_2nd_term}. Break the
integral into two parts,
\begin{align}
    \nonumber
    \int_0^t g(u)^2 \int_0^u h(s) \dd s \dd u  &= 
        \int_0^{t_i} g(u)^2 \int_0^u h(s) \dd s \dd u + 
        \int_{t_i}^t g(u)^2 \int_0^u h(s) \dd s \dd u \\
        \nonumber
        &= M_i + \int_{t_i}^t g_i^2 \left[
                H_i + h_i( u - t_i)
            \right] \dd u \\
        \nonumber
        &= M_i +
        g_i^2 H_i \Delta t_i + \half g_i^2 h_i (t^2 - t_i^2)
        - g_i^2 h_i t_i \Delta t_i \\
        \nonumber
        &=M_i + g_i^2 H_i \Delta t_i + \half g_i^2 h_i \Delta t_i^2 \\
        &= M_i + H_i \Delta G_i + \half \Delta H_i \Delta G_i
\end{align}
where
\begin{align}
    M_i \coloneqq \int_0^{t_i} g(u)^2 \int_0^u h(s) \dd s \dd u 
\end{align}
Substitute to evaluate $A(t, T)$, we have
\begin{align}
    A(t, T) &= (G_i H_i - M_i) + G_i \Delta H_i + \half \Delta H_i \Delta G_i, \quad t_i < t < t_{i+1}
\end{align}
Note $G_i$, $H_i$, and $M_i$ are quantities with integration up to the integer point (no fractions). 

\subsubsection{Expression of B(t, T)}
\begin{align}
    B(t, T) &= B(0, T) - B(0, t)\\
    B(0, t) &= \int_0^t h(u) \dd u = H_i + h_i\Delta t_i, \quad t_i < t < t_{i+1}
\end{align}


\end{appendices}


\end{document}